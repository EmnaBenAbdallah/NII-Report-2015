% Process Hitting

\section{Process Hitting }

Definition \ref{def:PH} introduces the Process Hitting(PH) \cite{PMR10-TCSB}
which allows to model a finite number of local levels,
called \emph{processes},
grouped into a finite set of components, called \emph{sorts}.
A process is noted $a_i$, where $a$ is the sort's name,
and $i$ is the process identifier within sort $a$.
At any time, exactly one process of each sort is \emph{active},
and the set of active processes is called a \emph{state}.

The concurrent interactions between processes are defined by a set of \emph{actions}.
Each action is responsible for the replacement of one process by another of the same sort
conditioned by the presence of at most one other process in the current state.
An action is denoted by $\PHfrappe{a_i}{b_j}{b_k}$, which is read as
“$a_i$ \emph{hits} $b_j$ to make it \emph{bounce} to $b_k$”,
where $a_i$, $b_j$, $b_k$ are processes of sorts $a$ and $b$,
called respectively \emph{hitter}, \emph{target} and
\emph{bounce} of the action.
We also call a \emph{self-hit} any action whose hitter and target sorts are the same,
that is, of the form: $\PHfrappe{a_i}{a_i}{a_k}$.

The PH is therefore a restriction of synchronous automata, where each transition
changes the local state of exactly one automaton,
and is triggered by the local states of at most two distinct automata.
This restriction in the form of the actions was chosen to permit
the development of efficient static analysis methods
based on abstract interpretation \cite{PMR12-MSCS}.

\begin{definition}[Process Hitting]\label{def:PH}
  A \emph{Process Hitting} is a triple $(\PHs,\PHl,\PHa)$ where:
  \begin{itemize}
    \item  $\PHs = \{a,b,\dots\}$ is the finite set of \emph{sorts};
    \item  $\PHl = \prod_{a\in\PHs} \PHl_a$ is the set of \emph{states} where
      $\PHl_a = \{a_0,\dots,a_{l_a}\}$
      is the finite set of \emph{processes} of sort $a\in\Sigma$
      and $l_a$ is a positive integer, with $a\neq b\Rightarrow \PHl_a \cap \PHl_b = \emptyset$;
    \item  $\PHa = \{ \PHfrappe{a_i}{b_j}{b_k} \in \PHl_a \times \PHl_b^2 \mid
      (a,b) \in \PHs^2 \wedge b_j\neq b_k \wedge a=b\Rightarrow a_i=b_j\}$
      is the finite set of \emph{actions}.
  \end{itemize}
\end{definition}

\begin{example*}
The figure \ref{fig:ph} represents a $\PH$ $(\PHs,\PHl,\PHa)$ with three sorts
($\PHs = \{a, b, c\}$) and:
$\PHl_a = \{a_0, a_1\}$,
$\PHl_b = \{b_0, b_1\}$,
$\PHl_z = \{z_0, z_1, z_2\}$.
\begin{figure}[ht]
\label{fig:ph} 
\centering
\begin{tikzpicture}%[font=\scriptsize]
%\path[use as bounding box] (0,-1) rectangle (4,4);

\TSort{(0,0)}{z}{3}{l}
\TSort{(2,4)}{b}{2}{t}
\TSort{(4,1)}{a}{2}{r}
\THit{b_0}{}{z_1}{.east}{z_2}
\THit{b_1}{}{z_0}{.north east}{z_2}
\THit{a_0}{}{b_1}{.south}{b_0}
\THit{a_1}{out=60,in=0,selfhit}{a_1}{.east}{a_0}

\path[bounce,bend right]
\TBounce{z_1}{}{z_2}{.south}
\TBounce{z_0}{bend right=50}{z_2}{.south east}
;
\path[bounce,bend left]
\TBounce{a_1}{}{a_0}{.north}
\TBounce{b_1}{}{b_0}{.south}
;

 \THit{z_0}{}{a_0}{.west}{a_1} 

\path[bounce,bend left]
\TBounce{a_0}{}{a_1}{.south}
;
\TState{a_0,b_0,z_1}
\end{tikzpicture}
\caption{
A PH model example with three sorts: $a$, $b$ and $z$ ($a$ is either at level 0 or 1, $b$ at either level 0 or 1 and $z$ at either level 0, 1 or 2). Boxes represent the \emph{sorts} (network components), circles represent the \emph{processes} (component levels), and the 5 \emph{actions} that model the dynamic behavior are depicted by pairs of arrows in solid and dotted lines. The grayed processes stand for the possible initial state: $\PHstate{a_1, b_0, z_1}$.
}
\end{figure}
\end{example*}

A state of the networks is a set of active processes containing a single process of each sort.
The active process of a given sort $a \in \PHs$ in a state $s \in \PHl$
is noted $\PHget{s}{a}$.
For any given process $a_i$ we also note: $a_i \in s$ if and only if $\PHget{s}{a} = a_i$. The dynamic of the PH networks is satisfied thanks to the actions. Indeed, the transition from one state $s_1$ to its successor $s_2$ is done when there is a playable action (definition \ref{def:playableAction}) at $s_1$. After each transition only one sort, or one component, changes its level from one process to another.

%\begin{definition}[Action]
%\label{def:PhAction}
%An action is noted $\PHfrappe{a_i}{b_j}{b_k}$ where $a_i$ is a process
%of sort $a$ and $b_j$, $b_k$ two processes of sort $b$. When $a_i = b_j$ , such an action is %refered as a self-action and $a_i$
%is called a self-hitting process.
%\end{definition}

\begin{definition} [Playable action]
\label{def:playableAction}
Let $\PH = (\PHs,\PHl,\PHa)$ be a Process Hitting and $s \in \PHl$ a state of $PH$.
We say that the action $h = \PHfrappe{a_i}{b_j}{b_k} \in \PHa$
is \emph{playable in state $s$} if and only if
$a_i \in s$ and $b_j \in s$ (\ie $\PHget{s}{a} = a_i$ and $\PHget{s}{b}=b_j$).
The resulting state after playing $h$ in $s$
is called a \emph{successor} of $s$ and
is denoted by $(s \play h)$,
where $\PHget{(s \play h)}{b} = b_k$ and
$\forall c \in \PHs, c \neq b \Rightarrow \PHget{(s \play h)}{c}=\PHget{s}{c}$.
\end{definition}

In some dynamics it is crucial to have information about the delays between two events (two states in PH). The normal actions cannot show this information we just know that the state $s_2$ will be after $s_1$ in the next step but it is not possible to know how much time this transition takes time. We propose to add this delay in the action which is responsable of the transition between the two states. That means that this action needs to be played during a specific time so that the system doesn't change the state.

\begin{definition}[timed action]
\label{def:TimedAction}

\end{definition}

We note that during these years the Process Hitting framework was improved and we added new type of sorts like cooperative sorts and new actions like conjoint actions and actions with delay.

The PH was chosen for several reasons.
First, it is a general framework that,
although it was mainly used for biological networks,
allows to represent any kind of dynamical model,
and converters to several other representations are available (see section%~\ref{comparison}).
Although an efficient dynamical analysis already exists for this framework,
based on an approximation of the dynamics,
it is interesting to identify its limits
and compare them to the approached we present later in this paper.
Finally, the particular form of the actions in a PH model allow
to easily represent them in ASP,
with one fact per action, as described in the next section.
Other representations may have required supplementary complexity;
for instance, a labeling would be required
if actions could be triggered by a variable number of processes.


\begin{definition}[Conjoint action]
\label{def:ConjointAction}

\end{definition}



%Parler de l'asynchrone model

The PH was chosen for several reasons.
First, it is a general framework that,
although it was mainly used for biological networks,
allows to represent any kind of dynamical model,
and converters to several other representations are available (see section%~\ref{comparison}).
Although an efficient dynamical analysis already exists for this framework,
based on an approximation of the dynamics,
it is interesting to identify its limits
and compare them to the approached we present later in this paper.
Finally, the particular form of the actions in a PH model allow
to easily represent them in ASP,
with one fact per action, as described in the next section.
Other representations may have required supplementary complexity;
for instance, a labeling would be required
if actions could be triggered by a variable number of processes.


The rest of the paper focuses on the resolution of the previous issues
with the use of ASP.
The enumeration of all fixed points of a PH model will be tackled in
section%~\ref{fixpoint}
and the verification of a reachability property will be the subject of
section%~\ref{dynamics}.

