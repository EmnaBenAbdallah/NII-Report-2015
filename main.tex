\documentclass[a4paper,12pt,final,english]{article}
% Pour une impression recto verso, utilisez plutôt ce documentclass :
%\documentclass[a4paper,11pt,twoside,final]{article}

\usepackage[utf8]{inputenc}
\usepackage[T1]{fontenc}
\usepackage[french]{babel}
\usepackage[pdftex]{graphicx}
\usepackage[top=3cm, bottom=3cm, left=3cm, right=3cm]{geometry}
\usepackage{setspace}
\usepackage{hyperref}
\usepackage[french]{varioref}
\usepackage{amsmath}
\usepackage{amsfonts}
\usepackage{amssymb}
\frenchbsetup{StandardLists=true}
\usepackage{amsthm}

\theoremstyle{definition} 
\newtheorem{definition}{Definition}
\theoremstyle{definition} 
\newtheorem{example}{Example}
\theoremstyle{remark} 
\newtheorem{theorem}{Thaorem}

%%%%%%%%%%%%%%%%%%%%
\usepackage{listings}
% Définition du langage ASP
\lstdefinelanguage{ASP}{\^^M}
}

% Définition des styles de tous les listings du document
\lstset{language=ASP,
basicstyle=\small,
columns=flexible,
keywordstyle=\bfseries,
firstnumber=last
}
\renewcommand{\thelstnumber}{\the\value{lstnumber}}
%% fin définition

%%%%%%%%%%%%%%%%%%%%%%%%%%%%%%%%%%
\usepackage{enumerate} % Personnalisation de la numérotation des listes
\usepackage{url}     % Mise en forme + liens pour URLs
\usepackage{array}   % Tableaux évolués
\usepackage{comment}

\usepackage{prettyref}
\newrefformat{def}{Definition~\ref{#1}}
\newrefformat{fig}{Figure~\ref{#1}}
\newrefformat{pro}{Property~\ref{#1}}
\newrefformat{pps}{Proposition~\ref{#1}}
\newrefformat{lem}{Lemma~\ref{#1}}
\newrefformat{th}{Theorem~\ref{#1}}
\newrefformat{sec}{Section~\ref{#1}}
%\newrefformat{subsec}{Subsect.~\ref{#1}}
\newrefformat{suppl}{Appendix~\ref{#1}}
\newrefformat{eq}{Eq.~\eqref{#1}}
\def\pref{\prettyref}

\newtheorem*{example*}{Example}{\itshape}{}

\usepackage{tikz}
\newdimen\pgfex
\newdimen\pgfem
\usetikzlibrary{arrows,shapes,shadows,scopes}
\usetikzlibrary{positioning}
\usetikzlibrary{matrix}
\usetikzlibrary{decorations.text}
\usetikzlibrary{decorations.pathmorphing}

%\input{macros/macros}

%%%%%
% Macros générales
\def\Pint{\textsc{PINT}}


% Notations spécifiques à ce papier
\newcommand{\PHdirectpredec}[1]{\PHs^{-1}(#1)}
\newcommand{\PHpredec}[1]{\f{pred}(#1)}
\newcommand{\PHpredecgene}[1]{\f{reg}({#1})}
\newcommand{\PHpredeccs}[1]{\PHpredec{#1} \setminus \Gamma}

\tikzstyle{boxed ph}=[]
\tikzstyle{sort}=[fill=lightgray,rounded corners]
\tikzstyle{process}=[circle,draw,minimum size=15pt,fill=white,
font=\footnotesize,inner sep=1pt]
\tikzstyle{black process}=[process, fill=black,text=white, font=\bfseries]
\tikzstyle{gray process}=[process, draw=black, fill=lightgray]
\tikzstyle{current process}=[process, draw=black, fill=lightgray]
\tikzstyle{process box}=[white,draw=black,rounded corners]
\tikzstyle{tick label}=[font=\footnotesize]
\tikzstyle{tick}=[black,-]%,densely dotted]
\tikzstyle{hit}=[->,>=angle 45]
\tikzstyle{selfhit}=[min distance=30pt,curve to]
\tikzstyle{bounce}=[densely dotted,->,>=latex]
\tikzstyle{hl}=[font=\bfseries,very thick]
\tikzstyle{hl2}=[hl]
\tikzstyle{nohl}=[font=\normalfont,thin]


\tikzstyle{aS}=[every edge/.style={draw,->,>=stealth}]
\tikzstyle{Asol}=[draw,circle,minimum size=5pt,inner sep=0,node distance=1.5cm]
\tikzstyle{Aproc}=[draw,node distance=1.2cm]
\tikzstyle{Aobj}=[node distance=1.5cm]
\tikzstyle{Anos}=[font=\Large]

%\tikzstyle{AprocPrio}=[Aproc,double]
\tikzstyle{AsolPrio}=[Asol,double]
\tikzstyle{AprocPrio}=[Aproc,double]
\tikzstyle{aSPrio}=[aS,double]

% Commandes À FAIRE
%\usepackage{color} % Couleurs du texte
%\newcommand{\todo}[1]{\textcolor{red}{\textbf{[[#1]]}}}
%\newcommand{\TODO}{\todo{TODO}}

%%%%%
% Id est
%\newcommand{\ie}{\textit{i.e.} }
\newcommand{\ie}{i.e.\ }
\newcommand{\resp}{resp.\ }

% Césures
\hyphenation{pa-ra-me-tri-za-tion}
\hyphenation{pa-ra-me-tri-za-tions}

\input{macros/macros}
\input{macros/macros-ph}
\input{macros/tikzstyles2.tex}
\input{macros/macros-abstr}
\input{ex.tex}
%%%%%%%%%%%%%%%%%%%%%%%%%%%%%%%%%%

\newcommand{\reporttitle}{Report}     % Titre
\newcommand{\reportauthor}{Emna \textsc{Ben Abdallah}} % Auteur
\newcommand{\reportsubject}{Internship at NII} % Sujet
\newcommand{\HRule}{\rule{\linewidth}{0.5mm}}
\setlength{\parskip}{1ex} % Espace entre les paragraphes

\hypersetup{
    pdftitle={\reporttitle},%
    pdfauthor={\reportauthor},%
    pdfsubject={\reportsubject},%
    pdfkeywords={rapport} {vos} {mots} {clés}
}

\begin{document}
  %\include{title}
  \cleardoublepage % Dans le cas du recto verso, ajoute une page blanche si besoin
  \tableofcontents % Table des matières
  \sloppy          % Justification moins stricte : des mots ne dépasseront pas des paragraphes
  \cleardoublepage
  \section*{Acknowledgement}
\addcontentsline{toc}{section}{Acknowledgement}

The internship opportunity I had with National Institute od Informatics (NII) was a great chance for learning and professional development. Therefore, I consider myself as a very lucky individual as I was provided with an opportunity to be a part of it. I am also grateful for having a chance to meet so many wonderful people and professionals who led me though this internship period.

I express my deepest thanks to Dr. Pr. Katsumi Inoue, my team leader and my supervisor for taking part in useful decision \& giving necessary advices and guidance and arranged all facilities to make life easier. I choose this moment to acknowledge his contribution gratefully.

Bearing in mind previous I am using this opportunity to express my deepest gratitude and special thanks to my Ph.D. supervisor Ass. Pr. Morgan Magnin who in spite of being extraordinarily busy with his duties, took time out to hear, guide and keep me on the correct path and allowing me to carry out my researches at their esteemed organization and extending during the training.

It is my radiant sentiment to place on record my best regards, deepest sense of gratitude to my supervisor Dr. P. Olivier Roux at France for his careful and precious guidance which were extremely valuable for my study both theoretically and practically. Also I want to thank him and thank Ass. P Morgan Magnin for their effort to allow me to do this interesting internship.

I also would like to thank all the people that worked in the office of NII in Tokyo. With their
patience and openness they created an enjoyable working environment.
Furthermore I want to thank all the scientist and students, with whom I did the fieldwork. We
experienced great things together.

I perceive as this opportunity as a big milestone in my career development. I will strive to use gained skills and knowledge in the best possible way, and I will continue to work on their improvement, in order to attain desired career objectives. Hope to continue cooperation with all of you in the future.


  \cleardoublepage
  \section*{Abstract}
\addcontentsline{toc}{section}{Abstract}

Almost the models of biological networks are not robust and need some times to be revised and adapted to the new observations. A system maintains its functions against internal and external perturbations, leading to topological changes in the network with varying delays. To understand the resilient behaviour of biological networks, we propose novel methods. First we propose an approach to model a time-dependent asynchronous and non-deterministic networks through Process Hitting (PH) framework which is a new framework particularly suitable to model biological regulatory networks. Second we have developed a novel network completion algorithm for time-varying networks to analyse its behavior based on the framework of network completion. This completion aims to make the minimum amount of modifications to a given network so that the resulting network is most consistent with the observed data.
We demonstrate the effectiveness of our proposed methods through computational experiments using synthetic gene expression data of the circadien clock network modeled through PH.  The results indicate that our methods exhibit good performance in terms of completing and inferring gene association networks with time-varying structures.

%This paper studies problems of completing a given Boolean network (Boolean circuit) so that the input/output behavior is consis- tent with given examples, where we only consider acyclic networks. These problems arise in the study of inference of signaling networks using re- porter proteins. We prove that these problems are


% We verify this by simulations of a simple model for interactions among cells.

%Circadian rhythms are certain periodic behaviours exhibited by living organism at different levels, including cellular and system-wide scales. Recent studies have found that the circa- dian rhythms of several peripheral organs in mammals, such as the liver, are able to entrain their clocks to received signals independent of other system level clocks, in particular when responding to signals generated during feeding. 
%The properties of the model show that the circadian system itself is strongly robust, and is able to continually evolve

%Asymptotic behavior is often of particular interest when an- alyzing asynchronous Boolean networks representing biological systems such as signal transduction or gene regulatory networks.

%In this paper, we propose a novel optimization-based method for computing all maximal symbolic steady states and motivate their use. In particular, we add a new result yielding a lower bound for the number of cyclic attractors and illustrate the methods with a short study of a MAPK pathway model.

  \cleardoublepage
  \include{intro}
  \cleardoublepage
  % Process Hitting

\section{Process Hitting }

Definition \ref{def:PH} introduces the Process Hitting(PH) \cite{PMR10-TCSB}
which allows to model a finite number of local levels,
called \emph{processes},
grouped into a finite set of components, called \emph{sorts}.
A process is noted $a_i$, where $a$ is the sort's name,
and $i$ is the process identifier within sort $a$.
At any time, exactly one process of each sort is \emph{active},
and the set of active processes is called a \emph{state}.

The concurrent interactions between processes are defined by a set of \emph{actions}.
Each action is responsible for the replacement of one process by another of the same sort
conditioned by the presence of at most one other process in the current state.
An action is denoted by $\PHfrappe{a_i}{b_j}{b_k}$, which is read as
“$a_i$ \emph{hits} $b_j$ to make it \emph{bounce} to $b_k$”,
where $a_i$, $b_j$, $b_k$ are processes of sorts $a$ and $b$,
called respectively \emph{hitter}, \emph{target} and
\emph{bounce} of the action.
We also call a \emph{self-hit} any action whose hitter and target sorts are the same,
that is, of the form: $\PHfrappe{a_i}{a_i}{a_k}$.

The PH is therefore a restriction of synchronous automata, where each transition
changes the local state of exactly one automaton,
and is triggered by the local states of at most two distinct automata.
This restriction in the form of the actions was chosen to permit
the development of efficient static analysis methods
based on abstract interpretation \cite{PMR12-MSCS}.

\begin{definition}[Process Hitting]\label{def:PH}
  A \emph{Process Hitting} is a triple $(\PHs,\PHl,\PHa)$ where:
  \begin{itemize}
    \item  $\PHs = \{a,b,\dots\}$ is the finite set of \emph{sorts};
    \item  $\PHl = \prod_{a\in\PHs} \PHl_a$ is the set of \emph{states} where
      $\PHl_a = \{a_0,\dots,a_{l_a}\}$
      is the finite set of \emph{processes} of sort $a\in\Sigma$
      and $l_a$ is a positive integer, with $a\neq b\Rightarrow \PHl_a \cap \PHl_b = \emptyset$;
    \item  $\PHa = \{ \PHfrappe{a_i}{b_j}{b_k} \in \PHl_a \times \PHl_b^2 \mid
      (a,b) \in \PHs^2 \wedge b_j\neq b_k \wedge a=b\Rightarrow a_i=b_j\}$
      is the finite set of \emph{actions}.
  \end{itemize}
\end{definition}

\begin{example*}
The figure \ref{fig:ph} represents a $\PH$ $(\PHs,\PHl,\PHa)$ with three sorts
($\PHs = \{a, b, c\}$) and:
$\PHl_a = \{a_0, a_1\}$,
$\PHl_b = \{b_0, b_1\}$,
$\PHl_z = \{z_0, z_1, z_2\}$.
\begin{figure}[ht]
\label{fig:ph} 
\centering
\begin{tikzpicture}%[font=\scriptsize]
%\path[use as bounding box] (0,-1) rectangle (4,4);

\TSort{(0,0)}{z}{3}{l}
\TSort{(2,4)}{b}{2}{t}
\TSort{(4,1)}{a}{2}{r}
\THit{b_0}{}{z_1}{.east}{z_2}
\THit{b_1}{}{z_0}{.north east}{z_2}
\THit{a_0}{}{b_1}{.south}{b_0}
\THit{a_1}{out=60,in=0,selfhit}{a_1}{.east}{a_0}

\path[bounce,bend right]
\TBounce{z_1}{}{z_2}{.south}
\TBounce{z_0}{bend right=50}{z_2}{.south east}
;
\path[bounce,bend left]
\TBounce{a_1}{}{a_0}{.north}
\TBounce{b_1}{}{b_0}{.south}
;

 \THit{z_0}{}{a_0}{.west}{a_1} 

\path[bounce,bend left]
\TBounce{a_0}{}{a_1}{.south}
;
\TState{a_0,b_0,z_1}
\end{tikzpicture}
\caption{
A PH model example with three sorts: $a$, $b$ and $z$ ($a$ is either at level 0 or 1, $b$ at either level 0 or 1 and $z$ at either level 0, 1 or 2). Boxes represent the \emph{sorts} (network components), circles represent the \emph{processes} (component levels), and the 5 \emph{actions} that model the dynamic behavior are depicted by pairs of arrows in solid and dotted lines. The grayed processes stand for the possible initial state: $\PHstate{a_1, b_0, z_1}$.
}
\end{figure}
\end{example*}

A state of the networks is a set of active processes containing a single process of each sort.
The active process of a given sort $a \in \PHs$ in a state $s \in \PHl$
is noted $\PHget{s}{a}$.
For any given process $a_i$ we also note: $a_i \in s$ if and only if $\PHget{s}{a} = a_i$. The dynamic of the PH networks is satisfied thanks to the actions. Indeed, the transition from one state $s_1$ to its successor $s_2$ is done when there is a playable action (definition \ref{def:playableAction}) at $s_1$. After each transition only one sort, or one component, changes its level from one process to another.

%\begin{definition}[Action]
%\label{def:PhAction}
%An action is noted $\PHfrappe{a_i}{b_j}{b_k}$ where $a_i$ is a process
%of sort $a$ and $b_j$, $b_k$ two processes of sort $b$. When $a_i = b_j$ , such an action is %refered as a self-action and $a_i$
%is called a self-hitting process.
%\end{definition}

\begin{definition} [Playable action]
\label{def:playableAction}
Let $\PH = (\PHs,\PHl,\PHa)$ be a Process Hitting and $s \in \PHl$ a state of $PH$.
We say that the action $h = \PHfrappe{a_i}{b_j}{b_k} \in \PHa$
is \emph{playable in state $s$} if and only if
$a_i \in s$ and $b_j \in s$ (\ie $\PHget{s}{a} = a_i$ and $\PHget{s}{b}=b_j$).
The resulting state after playing $h$ in $s$
is called a \emph{successor} of $s$ and
is denoted by $(s \play h)$,
where $\PHget{(s \play h)}{b} = b_k$ and
$\forall c \in \PHs, c \neq b \Rightarrow \PHget{(s \play h)}{c}=\PHget{s}{c}$.
\end{definition}

In some dynamics it is crucial to have information about the delays between two events (two states in PH). The normal actions cannot show this information we just know that the state $s_2$ will be after $s_1$ in the next step but it is not possible to know how much time this transition takes time. We propose to add this delay in the action which is responsable of the transition between the two states. That means that this action needs to be played during a specific time so that the system doesn't change the state.

\begin{definition}[timed action]
\label{def:TimedAction}

\end{definition}

We note that during these years the Process Hitting framework was improved and we added new type of sorts like cooperative sorts and new actions like conjoint actions and actions with delay.

The PH was chosen for several reasons.
First, it is a general framework that,
although it was mainly used for biological networks,
allows to represent any kind of dynamical model,
and converters to several other representations are available (see section%~\ref{comparison}).
Although an efficient dynamical analysis already exists for this framework,
based on an approximation of the dynamics,
it is interesting to identify its limits
and compare them to the approached we present later in this paper.
Finally, the particular form of the actions in a PH model allow
to easily represent them in ASP,
with one fact per action, as described in the next section.
Other representations may have required supplementary complexity;
for instance, a labeling would be required
if actions could be triggered by a variable number of processes.


\begin{definition}[Conjoint action]
\label{def:ConjointAction}

\end{definition}



%Parler de l'asynchrone model

The PH was chosen for several reasons.
First, it is a general framework that,
although it was mainly used for biological networks,
allows to represent any kind of dynamical model,
and converters to several other representations are available (see section%~\ref{comparison}).
Although an efficient dynamical analysis already exists for this framework,
based on an approximation of the dynamics,
it is interesting to identify its limits
and compare them to the approached we present later in this paper.
Finally, the particular form of the actions in a PH model allow
to easily represent them in ASP,
with one fact per action, as described in the next section.
Other representations may have required supplementary complexity;
for instance, a labeling would be required
if actions could be triggered by a variable number of processes.


The rest of the paper focuses on the resolution of the previous issues
with the use of ASP.
The enumeration of all fixed points of a PH model will be tackled in
section%~\ref{fixpoint}
and the verification of a reachability property will be the subject of
section%~\ref{dynamics}.


  \cleardoublepage
  \include{circadianClock}
  \cleardoublepage
  % PH through ASP

\section{PH through ASP}
\label{sec:ph-asp}
\subsection{Translation of PH networks to ASP}

\subsection{Circadian clock in ASP}
	
  \cleardoublepage
   \include{completionPH}
  \cleardoublepage
  \include{concl}
  \cleardoublepage
  
  \bibliographystyle{plain}
  \bibliography{biblio}
    
\end{document}

